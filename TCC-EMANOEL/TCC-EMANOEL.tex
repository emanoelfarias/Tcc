
%% abtex2-modelo-trabalho-academico.tex, v-1.9.6 laurocesar
%% Copyright 2012-2016 by abnTeX2 group at http://www.abntex.net.br/ 
%%
%% This work may be distributed and/or modified under the
%% conditions of the LaTeX Project Public License, either version 1.3
%% of this license or (at your option) any later version.
%% The latest version of this license is in
%%   http://www.latex-project.org/lppl.txt
%% and version 1.3 or later is part of all distributions of LaTeX
%% version 2005/12/01 or later.
%%
%% This work has the LPPL maintenance status `maintained'.
%% 
%% The Current Maintainer of this work is the abnTeX2 team, led
%% by Lauro César Araujo. Further information are available on 
%% http://www.abntex.net.br/
%%
%% This work consists of the files abntex2-modelo-trabalho-academico.tex,
%% abntex2-modelo-include-comandos and abntex2-modelo-references.bib
%%

% ------------------------------------------------------------------------
% ------------------------------------------------------------------------
% abnTeX2: Modelo de Trabalho Academico (tese de doutorado, dissertacao de
% mestrado e trabalhos monograficos em geral) em conformidade com 
% ABNT NBR 14724:2011: Informacao e documentacao - Trabalhos academicos -
% Apresentacao
% ------------------------------------------------------------------------
% ------------------------------------------------------------------------

\documentclass[
	% -- opções da classe memoir --
	12pt,				% tamanho da fonte
	openany,			% capítulos começam em pág ímpar (insere página vazia caso preciso)
	oneside,			% para impressão em recto e verso. Oposto a oneside
	a4paper,			% tamanho do papel. 
	% -- opções da classe abntex2 --
	%chapter=TITLE,		% títulos de capítulos convertidos em letras maiúsculas
	%section=TITLE,		% títulos de seções convertidos em letras maiúsculas
	%subsection=TITLE,	% títulos de subseções convertidos em letras maiúsculas
	%subsubsection=TITLE,% títulos de subsubseções convertidos em letras maiúsculas
	% -- opções do pacote babel --
	english,			% idioma adicional para hifenização
	french,				% idioma adicional para hifenização
	spanish,			% idioma adicional para hifenização
	brazil				% o último idioma é o principal do documento
	]{abntex2}

% ---
% Pacotes básicos 
% ---
\usepackage{lmodern}			% Usa a fonte Latin Modern			
\usepackage[T1]{fontenc}		% Selecao de codigos de fonte.
\usepackage[utf8]{inputenc}		% Codificacao do documento (conversão automática dos acentos)
\usepackage{lastpage}			% Usado pela Ficha catalográfica
\usepackage{indentfirst}		% Indenta o primeiro parágrafo de cada seção.
\usepackage{color}				% Controle das cores
\usepackage{graphicx}			% Inclusão de gráficos
\usepackage{microtype} 			% para melhorias de justificação
% ---
		
% ---
% Pacotes adicionais, usados apenas no âmbito do Modelo Canônico do abnteX2
% ---
\usepackage{lipsum}				% para geração de dummy text
% ---

% ---
% Pacotes de citações
% ---
\usepackage[brazilian,hyperpageref]{backref}	 % Paginas com as citações na bibl
\usepackage[alf]{abntex2cite}	% Citações padrão ABNT

% --- 
% CONFIGURAÇÕES DE PACOTES
% --- 

% ---
% Configurações do pacote backref
% Usado sem a opção hyperpageref de backref
\renewcommand{\backrefpagesname}{Citado na(s) página(s):~}
% Texto padrão antes do número das páginas
\renewcommand{\backref}{}
% Define os textos da citação
\renewcommand*{\backrefalt}[4]{
	\ifcase #1 %
		Nenhuma citação no texto.%
	\or
		Citado na página #2.%
	\else
		Citado #1 vezes nas páginas #2.%
	\fi}%
% ---

% ---
% Informações de dados para CAPA e FOLHA DE ROSTO
% ---
\titulo{PROPOSTA DE SISTEMA DE BAIXO CUSTO PARA MENSURAR O TREMOR DURANTE O TESTE DE FINGER TAPPING EM PACIÊNTES COM PARKINSON USANDO TÉCNICAS DE PROCESSAMENTO DE IMAGENS}
\autor{Emanoel Anderson dos Santos Farias}
\local{Maceió}
\data{2018}
\orientador{Dr. Leonardo Medeiros}

%\instituicao{%
%  Universidade do Brasil -- UBr
% \par
%  Faculdade de Arquitetura da Informação
%  \par
%  Programa de Pós-Graduação}
	
\tipotrabalho{Monografia (Trabalho de Conclusão de Curso)}
% O preambulo deve conter o tipo do trabalho, o objetivo, 
% o nome da instituição e a área de concentração 
\preambulo{Trabalho de Conclusão de Curso apresentado ao Curso de Bacharelado em Sistemas de Informação do Instituto Federal de Educação, Ciência e Tecnologia de Alagoas como requisito parcial para obtenção do título de bacharel em Sistemas de Informação.}
% ---


% ---
% Configurações de aparência do PDF final

% alterando o aspecto da cor azul
\definecolor{blue}{RGB}{41,5,195}

% informações do PDF
\makeatletter
\hypersetup{
     	%pagebackref=true,
		pdftitle={\@title}, 
		pdfauthor={\@author},
    	pdfsubject={\imprimirpreambulo},
	    pdfcreator={LaTeX with abnTeX2},
		pdfkeywords={abnt}{latex}{abntex}{abntex2}{trabalho acadêmico}, 
		colorlinks=true,       		% false: boxed links; true: colored links
    	linkcolor=blue,          	% color of internal links
    	citecolor=blue,        		% color of links to bibliography
    	filecolor=magenta,      		% color of file links
		urlcolor=blue,
		bookmarksdepth=4
}
\makeatother
% --- 

% --- 
% Espaçamentos entre linhas e parágrafos 
% --- 

% O tamanho do parágrafo é dado por:
\setlength{\parindent}{1.3cm}

% Controle do espaçamento entre um parágrafo e outro:
\setlength{\parskip}{0.2cm}  % tente também \onelineskip

% ---
% compila o indice
% ---
\makeindex
% ---

% ----
% Início do documento
% ----
\begin{document}

% Seleciona o idioma do documento (conforme pacotes do babel)
%\selectlanguage{english}
\selectlanguage{brazil}

% Retira espaço extra obsoleto entre as frases.
\frenchspacing 

% ----------------------------------------------------------
% ELEMENTOS PRÉ-TEXTUAIS
% ----------------------------------------------------------
% \pretextual

% ---
% Capa
% ---

\begin{center}

\textsf{\textsc{Instituto Federal de Educação, Ciência e Tecnologia de Alagoas\\
 Campus Maceió\\
 Coordenação de Informática \\
 Curso Superior de Bacharelado em Sistemas de Informação 
}} 

\end{center}


\imprimircapa
% ---

% ---
% Folha de rosto
% (o * indica que haverá a ficha bibliográfica)
% ---
\imprimirfolhaderosto*
% ---

% ---
% Inserir a ficha bibliografica
% ---

% Isto é um exemplo de Ficha Catalográfica, ou ``Dados internacionais de
% catalogação-na-publicação''. Você pode utilizar este modelo como referência. 
% Porém, provavelmente a biblioteca da sua universidade lhe fornecerá um PDF
% com a ficha catalográfica definitiva após a defesa do trabalho. Quando estiver
% com o documento, salve-o como PDF no diretório do seu projeto e substitua todo
% o conteúdo de implementação deste arquivo pelo comando abaixo:
%
% \begin{fichacatalografica}
%     \includepdf{fig_ficha_catalografica.pdf}
% \end{fichacatalografica}

%\begin{fichacatalografica}
%	\sffamily
%	\vspace*{\fill}					% Posição vertical
%	\begin{center}					% Minipage Centralizado
%	\fbox{\begin{minipage}[c][8cm]{13.5cm}		% Largura
%	\small
%	\imprimirautor
	%Sobrenome, Nome do autor
	
%	\hspace{0.5cm} \imprimirtitulo  / \imprimirautor. --
%	\imprimirlocal, \imprimirdata-
	
%	\hspace{0.5cm} \pageref{LastPage} p. : il. (algumas color.) ; 30 cm.\\
	
%	\hspace{0.5cm} \imprimirorientadorRotulo~\imprimirorientador\\
	
%	\hspace{0.5cm}
%	\parbox[t]{\textwidth}{\imprimirtipotrabalho~--~\imprimirinstituicao,
%	\imprimirdata.}\\
	
%	\hspace{0.5cm}
%		1. Palavra-chave1.
%		2. Palavra-chave2.
%		2. Palavra-chave3.
%		I. Orientador.
%		II. Universidade xxx.
%		III. Faculdade de xxx.
%		IV. Título 			
%	\end{minipage}}
%	\end{center}
%\end{fichacatalografica}
% ---

% ---
% Inserir errata
% ---
%\begin{errata}
%Elemento opcional da \citeonline[4.2.1.2]{NBR14724:2011}. Exemplo:

%\vspace{\onelineskip}

%FERRIGNO, C. R. A. \textbf{Tratamento de neoplasias ósseas apendiculares com
%reimplantação de enxerto ósseo autólogo autoclavado associado ao plasma
%rico em plaquetas}: estudo crítico na cirurgia de preservação de membro em
%cães. 2011. 128 f. Tese (Livre-Docência) - Faculdade de Medicina Veterinária e
%Zootecnia, Universidade de São Paulo, São Paulo, 2011.

%\begin{table}[htb]
%\center
%\footnotesize
%\begin{tabular}{|p{1.4cm}|p{1cm}|p{3cm}|p{3cm}|}
%  \hline
%   \textbf{Folha} & \textbf{Linha}  & \textbf{Onde se lê}  & \textbf{Leia-se}  \\
%    \hline
%   1 & 10 & auto-conclavo & autoconclavo\\
%  \hline
%\end{tabular}
%\end{table}

%\end{errata}
% ---

% ---
% Inserir folha de aprovação
% ---

% Isto é um exemplo de Folha de aprovação, elemento obrigatório da NBR
% 14724/2011 (seção 4.2.1.3). Você pode utilizar este modelo até a aprovação
% do trabalho. Após isso, substitua todo o conteúdo deste arquivo por uma
% imagem da página assinada pela banca com o comando abaixo:
%
% \includepdf{folhadeaprovacao_final.pdf}
%
\begin{folhadeaprovacao}

  \begin{center}
    {\ABNTEXchapterfont\large\imprimirautor}

    \vspace*{\fill}\vspace*{\fill}
    \begin{center}
      \ABNTEXchapterfont\bfseries\Large\imprimirtitulo
    \end{center}
    \vspace*{\fill}
    
    \hspace{.45\textwidth}
    \begin{minipage}{.5\textwidth}
        \imprimirpreambulo
    \end{minipage}%
    \vspace*{\fill}
   \end{center}
        
   Trabalho aprovado. \imprimirlocal, XX de XXXXXXX de 2018:

   \assinatura{\textbf{\imprimirorientador} \\ Orientador} 
   \assinatura{\textbf{Professor} \\ Convidado 1}
   \assinatura{\textbf{Professor} \\ Convidado 2}
   %\assinatura{\textbf{Professor} \\ Convidado 3}
   %\assinatura{\textbf{Professor} \\ Convidado 4}
      
   \begin{center}
    \vspace*{0.5cm}
    {\large\imprimirlocal}
    \par
    {\large\imprimirdata}
    \vspace*{1cm}
  \end{center}
  
\end{folhadeaprovacao}
% ---

% ---
% Dedicatória
% ---
%\begin{dedicatoria}
%   \vspace*{\fill}
%   \centering
%   \noindent
%   \textit{ Este trabalho é dedicado às crianças adultas que,\\
%   quando pequenas, sonharam em se tornar cientistas.} \vspace*{\fill}
%\end{dedicatoria}
% ---

% ---
% Agradecimentos
% ---
\begin{agradecimentos}

Agradeço a todos que me deram forças para seguir em frente, em especial minha amanda esposa Fernanda, meus pais, Antônio e Maria Aparecida e minhas queridas irmãs. Agradeço também ao meu orientador Dr. Leonardo Medeiros pela grande força e incentivo a construção deste trabalho. Sou grato também a todos da coordenação do cursos de Sistemas de Informação e ao Instituto Federal de Alagoas de maneira geral. 

\end{agradecimentos}
% ---

% ---
% Epígrafe
% ---
\begin{epigrafe}
    \vspace*{\fill}
	\begin{flushright}
		\textit{Cogito, ergo sum. (René Descarte)}
	\end{flushright}
\end{epigrafe}
% ---

% ---
% RESUMOS
% ---

% resumo em português
\setlength{\absparsep}{18pt} % ajusta o espaçamento dos parágrafos do resumo
\begin{resumo}

Cerca de 7,5 milhões de pessoas no mundo inteiro tem a doença de Parkinson, aproximadamente 200 mil apenas no Brasil
segundo dados da Organização Mundial da Saúde (OMS). O Parkinson é uma doença crônica neurodegenerativa tendo como
principais sintomas a a bradicinesia, tremor, rigidez e problemas posturais. Desta maneira, sistemas e testes que venham quantificar e avaliar o funcionamento das articulações do corpo são de grande importância para o diagnóstico e avaliação desta doença. Dentre os testes realizados para avaliação e diagnóstico encontra-se o finger tapping, ou batida de dedos, que consiste em avaliar o controle motor fino de pacientes acometidos ao parkinson e é descrito na Escala Unificada de Avaliação da Doença de Parkinson (UPDRS), neste o especialista realiza a medição por meio de uma observação e interpretação subjetiva do tremor entre o movimentar dos dedos do paciente. Na literatura e na industria estão são encontrados diversos sistemas para mensurar o tremor durante o teste de finger tapping, com o uso de sensores como acelerômetros e giroscópios e câmeras estereoscópicas, no entanto, esses sistemas são de custos elevados e muitas vezes usam diversos elementos conectados à mão do paciente acometido, o que pode alterar os valores medidos. Por tanto, é de grande valia um sistema de baixo custo, de fácil uso, e que não interfira da medição, que consiga precisar as variáveis observadas pelo especialista afim de tornar a avaliação ou diagnóstico eficientes. O objetivo principal propor um sistema de baixo custo que consiga aferir objetivamente o tremor durante os teste de finger tapping usando processamento digital de imagens. A metodologia usada foi do tipo ... Os resultados obtidos mostram que é possível mensurar com excelente precisão os tremores dos dedos utilizando o sistema proposto.


\textbf{Palavras-chave}: Parkinson. Finger Tapping Test. UPDRS. Processamento de imagem.
\end{resumo}

% resumo em inglês
\begin{resumo}[Abstract]
 \begin{otherlanguage*}{english}
 
About 7.5 million people worldwide have Parkinson's disease, approximately 200,000 in Brazil alone
according to data from the World Health Organization (WHO). Parkinson's disease is a chronic neurodegenerative disease
bradykinesia, tremor, stiffness and postural problems. In this way, systems and tests that quantify and evaluate the functioning of the joints of the body are of great importance for the diagnosis and evaluation of this disease. Among the tests performed for evaluation and diagnosis is finger tapping, which consists of evaluating the fine motor control of patients with Parkinson's disease and is described in the Unified Parkinson's Disease Assessment Scale (UPDRS), in this study. the specialist performs the measurement by means of a subjective observation and interpretation of the tremor between the moving of the patient's fingers. In literature and industry are found several systems to measure the tremor during the finger tapping test, with the use of sensors such as accelerometers and gyroscopes and stereoscopic cameras, however, these systems are expensive and often use several connected elements the affected patient, which can change the measured values. Therefore, a low-cost, easy-to-use system that does not interfere with measurement is helpful, which is able to specify the variables observed by the specialist in order to make the evaluation or diagnosis efficient. The main objective is to propose a low cost system that can objectively measure the tremor during finger tapping tests using digital image processing. The methodology used was of the type ... The results obtained show that it is possible to measure with excellent accuracy the tremors of the fingers using the proposed system.

   \vspace{\onelineskip}
 
   \noindent 
   \textbf{Keywords}: Parkinson. Finger Tapping Test. UPDRS. Image processing.
 \end{otherlanguage*}
\end{resumo}

% resumo em francês 
%\begin{resumo}[Résumé]
% \begin{otherlanguage*}{french}
%    Il s'agit d'un résumé en français.
% 
%   \textbf{Mots-clés}: latex. abntex. publication de textes.
% \end{otherlanguage*}
%\end{resumo}

% resumo em espanhol
%\begin{resumo}[Resumen]
% \begin{otherlanguage*}{spanish}
%  Este es el resumen en español.
% 
%  \textbf{Palabras clave}: latex. abntex. publicación de textos.
% \end{otherlanguage*}
%\end{resumo}
% ---

% ---
% inserir lista de ilustrações
% ---
\pdfbookmark[0]{\listfigurename}{lof}
\listoffigures*
\cleardoublepage
% ---

% ---
% inserir lista de tabelas
% ---
\pdfbookmark[0]{\listtablename}{lot}
\listoftables*
\cleardoublepage
% ---

% ---
% inserir lista de abreviaturas e siglas
% ---
\begin{siglas}
  \item[ABNT] Associação Brasileira de Normas Técnicas
  \item[abnTeX] ABsurdas Normas para TeX
\end{siglas}
% ---

% ---
% inserir lista de símbolos
% ---
\begin{simbolos}
  \item[$ \Gamma $] Letra grega Gama
  \item[$ \Lambda $] Lambda
  \item[$ \zeta $] Letra grega minúscula zeta
  \item[$ \in $] Pertence
\end{simbolos}
% ---

% ---
% inserir o sumario
% ---
\pdfbookmark[0]{\contentsname}{toc}
\tableofcontents*
\cleardoublepage
% ---



% ----------------------------------------------------------
% ELEMENTOS TEXTUAIS
% ----------------------------------------------------------
\textual

% ----------------------------------------------------------
% Introdução (exemplo de capítulo sem numeração, mas presente no Sumário)
% ----------------------------------------------------------
\chapter*[Introdução]{Introdução}
%\addcontentsline{toc}{chapter}{Introdução}
% ----------------------------------------------------------

DIGITAR AQUI A INTRODUÇÃO



% ---
% Capitulo de revisão de literatura
% ---
\chapter{Revisão da literatura}
% ---

% ---
\section{A doença de Parkinson}
% ---
Em 1817 James Parkinson deu o passo inicial sobre a DP em sua monografia \textit{An essay on the shaking palsy}, onde denominou o distúrbio de paralisia agitante, nesse trabalho analisou e descreveu as condições sobre as quais ocorriam a doença de maneira precisa, e, mais tarde, Jean-Martin Charcot, um eminente estudioso do tema e o primeiro a propor um tratamento medicamentoso para a DP, cunhou do termo doença de Parkinson em homenagem a sua pesquisa. Em 1929 Kinnier Wilson também destaca-se neste campo de estudos, sendo o primeiro a descrever a acinesia (redução da quantidade de movimentos). Após estes trabalhos ocorreram diversas descobertas, como a observação das modificações da substância negra em 1919, os primeiros medicamentos entre 1950 e 1960, descoberta da relação da dopamina com a DP no início dos anos 60 e adequação dos modelos experimentais para pesquisa na área em 1980 \cite{limongi2001,ebook2016}.

A doença de Parkinson (DP) é uma patologia neurodegenerativa crônica e de caráter progressivo e ainda incurável que afeta principalmente o sistema motor causando rigidez muscular, tremor, alteração na postura e bradicinesia (lentidão ao executar movimentos) e acinesia (redução da quantidade de movimento). Além disso, a DP pode causar também outras manifestações não motoras como depressão, problemas no sono e distúrbios relacionados à memória \cite{limongi2001}. O termo Parkinson não deve ser confundido com Parkinsonismo, de acordo com \citeonline{ebook2016} este diz respeito a um grupo de doenças neurológicas que causam lentidão ao movimentar-se, como o Alzheimer ou alguma lesão cerebral, enquanto aquele refere-se ao tipo mais comum de Parkinsonismo, denominado parkinsonismo primário.

Outra característica a salientar sobre a DP é que esta avança gradualmente e lentamente ao longo do tempo, progredindo em vários estágios, inicialmente afetando apenas um lado do corpo e em estágios mais avançados praticamente todas articulações \cite{rewar2015}, desta maneira, métodos de avaliação deste transtorno que consigam perceber objetivamente pequenas melhoras ou evolução da doença ao longo dos anos tornam-se imprescindíveis para uma melhor compreensão da DP.


\subsection{Diagnóstico}

O diagnóstico da doença de Parkinson não é trivial uma vez que a doença progride lentamente e em vários estágios e as variedades de patologias semelhantes ao Parkinson são enormes, o que dificulta o diagnóstico em estágios mais iniciais do problema \cite{ebook2016}. Excetuando-se os exames de escaneamento computadorizado utilizando marcadores, o diagnóstico é baseado em critérios clínicos que, basicamente, consistem na avaliação do histórico do paciente e uma avaliação cuidadosa em sintomas motores como a bradicinesia, rigidez e congelamento do movimento e instabilidade postural, sendo este último uma manifestação mais avançada da doença \cite{kalia2015}.

Os sintomas da DP são inúmeros e de diversas origens, o que dificulta ainda mais o diagnóstico, para \citeonline{rewar2015} os sintomas não motores são mais difíceis ainda de serem observados e interpretados, como por exemplo a dor, depressão, problemas de memória, distúrbios do sono e alteração inicial da voz. \citeonline{moreira2007} considera seis características motoras clínicas básicas importantes ao diagnóstico da DP: tremor em repouso a uma frequência de 4-5hz, rigidez, anormalidades posturais, a bradicinesia, perda dos reflexos e o congelamento, ressaltando a ocorrência de distúrbios durante o sono REM, anormalidades na olfação e disfunção autonômica como constipação e alteração da pressão arterial. Fatores genéticos também devem ser observados. Segundo \citeonline{kalia2015}, o paciente tiver parentes diretos acometidos a DP o teste genético podem ajudar no diagnóstico, porém mesmo o teste resultando em positivo para DP, isso não fornece diagnóstico precoce definitivo, devida a complexidade da doença. 

Existem diversos exames e testes que podem ser utilizados para o auxílio ao diagnóstico:

\begin{itemize}
	\item MRI Scan (Magnetic Resonance Imaging)
	\item CT scan (Computerized Tomography)
	\item DaT Scan (Dopamine transporter Scan)
	\item Metaiodobenzylguanidine (MIBG) scan:
	\item
\end{itemize}

Para um diagnóstico mais preciso da DP é imprescindível que não seja levado em consideração apenas exames neurológicos ou genéticos, \citeonline{barbosa2005} argumenta que a avaliação clássica da DP é feita avaliando o histórico do paciente, seu exame neurológico e a resposta à terapia dopaminérgica, esse última consistindo em fazer com que o paciente ingira medicamentos com a função de suprir a falta de dopamina e observar se há uma melhora que, neste sentido, elevariam as chances de se tratar da DP já que ouve uma resposta positiva a drogas de ação antiparkinsoniana.

Um proposta para realização do diagnóstico por meio da análise motora e anamnese é inicialmente identificar se a doença trata-se de uma síndrome parkinsoniana dividindo-a em dois tipos básico, a forma rígido-acinética, caracterizada pela presença de acinesia ou rigidez e a forma hipercinética havendo apenas o tremor e logo após identificar o parkinsonismo primário (DP), isso é feito excluindo-se a possibilidade do paciente ter parkinsonismos secundário ou atípico, e essa exclusão é realizada analisando as causas que geralmente estão relacionadas a esses tipos de parkinsonismo \cite{barbosa2005}.

Os níveis de tremor são características observáveis, e por tanto relativa de cada especialista, para que haja uma distinção entre as diversas variações de parkinsonismo é importante usar descrições padrões para os tipos de tremores, ou exemplo é o Quadro 1 criado pela \textit{Movement Disorders Society}, que é utilizado para uma análise mais detalhada em função da semelhança entre as doenças \cite{barbosa2005}.


\subsection{Principais Sintomas}
\subsection{Causas da Doença}

Ainda não há compreensão completa das causas da DP, ou seja, apresenta uma etiologia idiopática, porém sabe-se que está fortemente relacionada com a morte de células produtoras de dopamina, que é um neutro transmissor fundamental para o controle dos movimentos, em uma área do mesencéfalo denominada substância negra e que quando os primeiros sintomas começam a tornar-se perceptíveis a substância negra já perdeu 60\% das células produtoras de dopamina \cite{moreira2007}.

\subsection{Estágios}
\subsection{Tratamento}
\subsection{Escalas de Avaliação}

\subsection{Finger Tapping}

% ----------------------------------------------------------
% PARTE
% ----------------------------------------------------------
\part{Resultados}
% ----------------------------------------------------------

% ---
% primeiro capitulo de Resultados
% ---
\chapter{Lectus lobortis condimentum}
% ---

% ---
\section{Vestibulum ante ipsum primis in faucibus orci luctus et ultrices
posuere cubilia Curae}
% ---

\lipsum[21-22]

% ---
% segundo capitulo de Resultados
% ---
\chapter{Nam sed tellus sit amet lectus urna ullamcorper tristique interdum
elementum}
% ---

% ---
\section{Pellentesque sit amet pede ac sem eleifend consectetuer}
% ---

\lipsum[24]

% ----------------------------------------------------------
% Finaliza a parte no bookmark do PDF
% para que se inicie o bookmark na raiz
% e adiciona espaço de parte no Sumário
% ----------------------------------------------------------
\phantompart

% ---
% Conclusão
% ---
\chapter{Conclusão}
% ---

\lipsum[31-33]

% ----------------------------------------------------------
% ELEMENTOS PÓS-TEXTUAIS
% ----------------------------------------------------------
\postextual
% ----------------------------------------------------------

% ----------------------------------------------------------
% Referências bibliográficas
% ----------------------------------------------------------
\bibliography{abntex2-modelo-references}

% ----------------------------------------------------------
% Glossário
% ----------------------------------------------------------
%
% Consulte o manual da classe abntex2 para orientações sobre o glossário.
%
%\glossary

% ----------------------------------------------------------
% Apêndices
% ----------------------------------------------------------

% ---
% Inicia os apêndices
% ---
\begin{apendicesenv}

% Imprime uma página indicando o início dos apêndices
\partapendices

% ----------------------------------------------------------
\chapter{Quisque libero justo}
% ----------------------------------------------------------

\lipsum[50]

% ----------------------------------------------------------
\chapter{Nullam elementum urna vel imperdiet sodales elit ipsum pharetra ligula
ac pretium ante justo a nulla curabitur tristique arcu eu metus}
% ----------------------------------------------------------
\lipsum[55-57]

\end{apendicesenv}
% ---


% ----------------------------------------------------------
% Anexos
% ----------------------------------------------------------

% ---
% Inicia os anexos
% ---
\begin{anexosenv}

% Imprime uma página indicando o início dos anexos
\partanexos

% ---
\chapter{Morbi ultrices rutrum lorem.}
% ---
\lipsum[30]

% ---
\chapter{Cras non urna sed feugiat cum sociis natoque penatibus et magnis dis
parturient montes nascetur ridiculus mus}
% ---

\lipsum[31]

% ---
\chapter{Fusce facilisis lacinia dui}
% ---

\lipsum[32]

\end{anexosenv}

%---------------------------------------------------------------------
% INDICE REMISSIVO
%---------------------------------------------------------------------
\phantompart
\printindex
%---------------------------------------------------------------------

\end{document}
